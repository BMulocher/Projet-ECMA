\documentclass[a4paper,11pt]{article}
\usepackage{amsmath}
\usepackage[francais]{babel}
\usepackage[utf8]{inputenc}
\usepackage[T1]{fontenc}
\usepackage{color}
\usepackage{graphicx}
\usepackage{amsfonts}
\usepackage{authblk}
\usepackage{listings}
\usepackage{subfigure}

%diverses options
%\setlength{\parindent}{0pt}  %pour enlever tous les alinéas
\usepackage[scale=0.80]{geometry} %pour ajuster les marges
\usepackage{enumitem} % pour que les enum soient avec des points

\usepackage{hyperref} % pour les liens cliquables

\usepackage{epstopdf} % images eps

%\usepackage{caption}

\title{MPRO - ECMA\\
Classement de territoires en zone de montagne}
\author{Adrien Mar\'echal\\Benoît Mulocher}
\date{\today}
\affil{CNAM}


\begin{document}

\maketitle
\tableofcontents
\newpage



\section{Rappel des notations et du problème}


Une commune est décrite par un ensemble de mailles $M$. Chaque maille $(i,j) \in M$ est affectée de :
\begin{itemize}
	\item un handicap de pente noté $H^p_{ij}$,
	\item un coefficient correcteur noté associé à la pente $C^p_{ij}$,
	\item un handicap d'altitude noté $H^a_{ij}$,
	\item un coefficient correcteur associé à l'altitude $C^a_{ij}$.
\end{itemize}
 ~\\

\`A l'ensemble $M$ des mailles sont associés un handicap de pente $H^p(M)$ et d'altitude $H^a(M)$, définis par :
\begin{equation*}
H^p(M) = \frac{\sum_{(i,j)\in M} H^p_{ij} C^p_{ij} S_{ij}}{\sum_{(i,j)\in M} C^p_{ij} S_{ij}} \quad \text{et} \quad H^a(M) = \frac{\sum_{(i,j)\in M} H^a_{ij} C^a_{ij} S_{ij}}{\sum_{(i,j)\in M} C^a_{ij} S_{ij}}
\end{equation*}
où $S_{ij}$ représente le pourcentage de la surface de la maille $(i,j)$ retenu.

Dans le cadre simplifié où une maille est soit retenue entièrement, soit non retenue, cela donne :
\begin{equation*}
H^p(M) = \frac{\sum_{(i,j)\in M} H^p_{ij} C^p_{ij} x_{ij}}{\sum_{(i,j)\in M} C^p_{ij} x_{ij}} \quad \text{et} \quad H^a(M) = \frac{\sum_{(i,j)\in M} H^a_{ij} C^a_{ij} x_{ij}}{\sum_{(i,j)\in M} C^a_{ij} x_{ij}}
\end{equation*}
où $x_{ij} = 1$ si la maille $(i,j)$ est retenue, et $0$ sinon.\\


Le problème s'énonce comme suit :\\

\textbf{Déterminer, pour l'ensemble M des mailles d'une commune donnée, la plus grande surface d'un seul tenant, c'est-à-dire le plus grand sous-ensemble $Z\subseteq M$ de mailles contigües, tel que la valeur de $H^p(Z) + H^a(Z)$ soit supérieure ou égale à 2.}


~\\
\newpage
\section{Modélisation du problème en ignorant la connexité}


\subsection{Modélisation du problème}

On cherche a trouver la plus grande surface vérifiant une certaine contrainte sur les handicaps de pente et d'altitude. La taille d'une telle surface est obtenue en comptant le nombre de maille retenues, et l'objectif est de maximiser cette taille. Dès lors, on peut écrire la fonction objectif comme :

\begin{equation}
\max_{x_{ij}, (i,j) \in M} \sum_{(i,j) \in M} x_{ij}
\end{equation}
où $x_{ij}$ sont des variables associées à chaque maille.\\


La contrainte $H^p(Z) + H^a(Z) \ge 2$, où $Z = \left\{ (i,j) \in M, x_{ij} = 1 \right\}$ s'écrit en fonction des variables $x_{ij}$ du problème, sous forme fractionnaire :
\begin{equation}
\frac{\sum_{(i,j)\in M} H^p_{ij} C^p_{ij} x_{ij}}{\sum_{(i,j)\in M} C^p_{ij} x_{ij}} + \frac{\sum_{(i,j)\in M} H^a_{ij} C^a_{ij} x_{ij}}{\sum_{(i,j)\in M} C^a_{ij} x_{ij}} \ge 2
\label{eq:contr_frac}
\end{equation}
où les $x_{ij}$ sont des booléens, comme annoncé dans la modélisation :
\begin{equation}
x_{ij} \in \left\{0,1 \right\} \qquad \forall (i,j) \in M
\end{equation}



Lorsque l'on ignore la contrainte de connexité, le problème complet s'écrit donc :
\begin{equation}\tag{$P_{frac}$}
\left\{ \begin{aligned}
& \max_{x_{ij}, (i,j) \in M} \sum_{(i,j) \in M} x_{ij} \\
& \qquad \begin{array}{cll}
\text{s.c.} & \frac{\sum_{(i,j)\in M} H^p_{ij} C^p_{ij} x_{ij}}{\sum_{(i,j)\in M} C^p_{ij} x_{ij}} + \frac{\sum_{(i,j)\in M} H^a_{ij} C^a_{ij} x_{ij}}{\sum_{(i,j)\in M} C^a_{ij} x_{ij}} \ge 2 & ~ \\
~ & x_{ij} \in \left\{0,1 \right\} &  ~~~~ \forall (i,j) \in M
\end{array} 
\end{aligned} \right.
\end{equation}




~\\
\subsection{Linéarisation de la contrainte fractionnaire}

Pour linéariser la contrainte fractionnaire, on introduit les variables $p_{ij}$ et $a_{ij}$, $(i,j) \in M$ qui vont remplacer
\begin{equation*}
p_{ij} = x_{ij} \sum_{(i,j)\in M} C^p_{ij} x_{ij}
\end{equation*}
et
\begin{equation*}
a_{ij} = x_{ij} \sum_{(i,j)\in M} C^a_{ij} x_{ij}
\end{equation*}


Après multiplication de la contrainte (\ref{eq:contr_frac}) par les dénominateurs, a on :
\begin{multline*}
\sum_{(i,j)\in M} \left( H^p_{ij} C^p_{ij} x_{ij} \left( \sum_{(i,j)\in M} C^a_{ij} x_{ij} \right) \right) \\
+ \sum_{(i,j)\in M} \left( H^a_{ij} C^a_{ij} x_{ij} \left( \sum_{(i,j)\in M} C^a_{ij} x_{ij} \right) \right)
\ge 2 \sum_{(i,j)\in M} \left( C^p_{ij} x_{ij} \left( \sum_{(i,j)\in M} C^a_{ij} x_{ij} \right) \right)
\end{multline*}
qu'on réécrit donc :
\begin{equation*}
\sum_{(i,j)\in M} H^p_{ij} C^p_{ij} a_{ij} + \sum_{(i,j)\in M}  H^a_{ij} C^a_{ij} p_{ij} \ge 2 \sum_{(i,j)\in M} C^p_{ij} a_{ij}
\end{equation*}
soit finalement :
\begin{equation}
\sum_{(i,j)\in M} \left( H^p_{ij} C^p_{ij} - 2 C^p_{ij} \right) a_{ij} + \sum_{(i,j)\in M}  H^a_{ij} C^a_{ij} p_{ij} \ge 0
\label{eq:contrainte}
\end{equation}


Pour que la formulation reste équivalente, il faut ajouter les contraintes suivantes sur les nouvelles variables $p_{ij}$ et $a_{ij}$, $(i,j) \in M$ :
\begin{align}
p_{ij} & \le x_{ij} \sum_{(i,j)\in M} C^p_{ij} \label{eq:a}\\
p_{ij} & \le \sum_{(i,j)\in M} C^p_{ij} x_{ij} \label{eq:b}\\
p_{ij} & \ge \sum_{(i,j)\in M} C^p_{ij} x_{ij} + (x_{ij}-1) \sum_{(i,j)\in M} C^p_{ij} \label{eq:c}\\
p_{ij} & \ge 0 \label{eq:d}\\
p_{ij} & \in \mathbb{R}
\end{align}
Les contraintes (\ref{eq:a}) et (\ref{eq:d}) assurent $p_{ij} = 0$ si $x_{ij} = 0$, tandis que les contraintes (\ref{eq:b}) et (\ref{eq:c}) assurent que $p_{ij}$ prenne bien la valeur attendue $\sum_{(i,j)\in M} C^p_{ij} x_{ij}$ quand $x_{ij} = 1$.


De la même manière pour les $a_{ij}$, $(i,j) \in M$ :
\begin{align}
a_{ij} & \le x_{ij} \sum_{(i,j)\in M} C^a_{ij} \\
a_{ij} & \le \sum_{(i,j)\in M} C^a_{ij} x_{ij} \\
a_{ij} & \ge \sum_{(i,j)\in M} C^a_{ij} x_{ij} + (x_{ij}-1) \sum_{(i,j)\in M} C^a_{ij} \\
a_{ij} & \ge 0 \\
a_{ij} & \in \mathbb{R}
\end{align}


\paragraph{Remarque :} Suite à ces changements, la contrainte "principale" (\ref{eq:contrainte}) n'est plus parfaitement symétrique en les termes associés à la pente et ceux associés à l'altitude. En particulier, rien n'empêche a priori que $H^p_{ij} C^p_{ij} - 2 C^p_{ij}$ (i.e. le coefficient affecté à $a_{ij}$) soit négatif pour certains $(i,j)$.

On aurait également pu inverser le rôle des $p_{ij}$ et des $a_{ij}$ dans la réécriture de la contrainte.












~\\
\newpage
\section{Modélisation de la connexité}

Pour modéliser la connexité, on a utilisé la méthode suggérée par l'énoncé. On définit des variables $d^h_{ij}$ pour représenter la distance $h \in \mathbb{N}$ d'une maille sélectionnée $(i,j)$ à une origine fixée. Ainsi, si on suppose qu'une origine a été déterminée, on aura :
\begin{equation*}
d^h_{ij} = \left\{
\begin{array}{cl}
1 & \quad \text{si la maille } (i,j) \text{ est sélectionnée et est à distance $h$ de l'origine} \\ 
0 & \quad \text{sinon}
\end{array} \right.
\end{equation*}


L'origine sera en fait une maille dont la distance $h$ est nulle. Comme on ne veut qu'une unique composante connexe, on détermine une origine par la contrainte 
\begin{equation}
\sum_{(i,j)\in M} d^0_{ij} = 1
\end{equation}

Le fait que l'on attribue une distance aux mailles retenues seules s'écrit
\begin{equation}
\sum_{h=0}^{h_{\max}} d^h_{ij} = x_{ij} \qquad \forall (i,j) \in M
\end{equation}
ce qui traduit le fait qu'il existe nécessairement un $h \in \mathbb{N}$ tel qu'une maille retenue soit à distance $h$ de l'origine. Cela implique également que l'origine doit bien être une maille sélectionnée.

$h_{\max}$ est ici un entier choisit suffisamment grand pour que toute maille soit à distance au plus $h_{\max}$ d'une autre maille. Typiquement, si l'ensemble des mailles $M$ s'inscrit dans un carré de taille $n\times m$, $h_{\max} = n m$ convient.\\


Enfin, une maille sélectionnée ne peut être à distance $h+1$ de l'origine, que si l'un de ses voisins au moins est lui même à distance $h$ de l'origine. On a ainsi :
\begin{equation}
d^{h+1}_{ij} \le d^h_{i+1~j} + d^h_{i-1~j} + d^h_{i~j+1} + d^h_{i~j-1} \qquad \forall h \in [\![ 0, h_{\max}-1 ]\!],~ \forall (i,j) \in M
\end{equation}



De cette façon, si $d^h_{ij} = 1$ pour une certaine maille $(i,j) \in M$, $h$ ne représente pas forcément la longueur du plus court chemin de l'origine à $(i,j)$, mais la taille d'un chemin existant. Seule nous important l'existence d'un tel chemin pour la connexité, cela n'est pas problématique.

\newpage
\section{Modélisation complète du problème en variables mixtes}



Le problème complet, sans contrainte fractionnaire et tenant compte de la connexité, s'écrit :

\begin{equation}\tag{$P$}\hspace*{-0.1\linewidth}
\left\{ \begin{aligned}
& \hspace{0.4\linewidth} \max_{x_{ij}, (i,j) \in M} \sum_{(i,j) \in M} x_{ij} \\
& ~ \\
& \begin{array}{clrc}
\text{s.c. } & \sum_{(i,j)\in M} \left( H^p_{ij} C^p_{ij} - 2 C^p_{ij} \right) a_{ij} + \sum_{(i,j)\in M}  H^a_{ij} C^a_{ij} p_{ij} \ge 0 & ~ & (1) \\
~ & ~ & ~ & ~ \\
~ & p_{ij} \le x_{ij} \sum_{(i,j)\in M} C^p_{ij} & \forall (i,j) \in M & (2) \\
~ & p_{ij} \le \sum_{(i,j)\in M} C^p_{ij} x_{ij} & \forall (i,j) \in M &  (3) \\
~ & p_{ij} \ge \sum_{(i,j)\in M} C^p_{ij} x_{ij} + (x_{ij}-1) \sum_{(i,j)\in M} C^p_{ij} & \forall (i,j) \in M & (4)\\
~ & p_{ij} \ge 0 & \forall (i,j) \in M & (5) \\
~ & ~ & ~ \\
~ & a_{ij} \le x_{ij} \sum_{(i,j)\in M} C^a_{ij} & \forall (i,j) \in M & (6) \\
~ & a_{ij} \le \sum_{(i,j)\in M} C^a_{ij} x_{ij} & \forall (i,j) \in M & (7) \\
~ & a_{ij} \ge \sum_{(i,j)\in M} C^a_{ij} x_{ij} + (x_{ij}-1) \sum_{(i,j)\in M} C^a_{ij} & \forall (i,j) \in M & (8)\\
~ & a_{ij} \ge 0 & \forall (i,j) \in M & (9)\\
~ & ~ & ~ & ~ \\
~ & \sum_{(i,j)\in M} d^0_{ij} = 1 & ~ & (10)\\
~ & \sum_{h=0}^{h_{\max}} d^h_{ij} = x_{ij} & \forall (i,j) \in M & (11)\\
~ & d^{h+1}_{ij} \le d^h_{i+1~j} + d^h_{i-1~j} + d^h_{i~j+1} + d^h_{i~j-1} & \forall (i,j) \in M, ~ \forall h \in [\![ 0, h_{\max}-1 ]\!] & (12)\\
~ & ~ & ~ \\
~ & x_{ij} \in \left\{ 0,1 \right\} & \forall (i,j) \in M \\
~ & p_{ij}, a_{ij} \in \mathbb{R} & \forall (i,j) \in M \\
~ & d^h_{ij} \in \left\{ 0,1 \right\} & \forall (i,j) \in M, ~ \forall h \in [\![ 0, h_{\max}-1 ]\!] \\
\end{array} 
\end{aligned} \right.
\end{equation}

\newpage

\section{Résolution du problème}

\subsection{Résolution par Cplex}

Pour la résolution via Cplex, nous avons utilisé l'API Cplex C++. Nous avons programmé deux solvers différents, un qui résout le problème complet et un qui résout le problème sans les contraintes de connexité.

La combinatoire du problème est telle que dès que les instances commencent à être de taille trop importante, les temps de calcul deviennent déraisonnables, même pour le solveur qui ne tient pas compte des contraintes de connexité.

\begin{table}[h!]
\begin{center}
\begin{tabular}{|l|c|c|c|c|c|c|c|c|c|c|}
\hline
Instances & 1 & 2 & 3 & 4 & 5 & 6 & 7 & 8 & 9 & 10  \\
\hline
5-8 (score)  & 15 & 4 & N/A & 18 & 25 & 13 & 18 & 14 & 9 & 16 \\
\hline
5-8 (temps en s)  & 1.0 & 0.0 & N/A & 1.0 & 0.0 & 2.0 & 0.0 & 1.0 & 1.0 & 0.0 \\
\hline
12-10 (temps en s)  & 1h + & N/A & N/A & N/A & N/A & N/A & N/A & N/A & N/A & N/A \\
\hline
\end{tabular}
\end{center}
\caption{Résultats du solveur Cplex}
\end{table}

A partir des instances de taille 12-10, le solveur Cplex met plus d'une heure à trouver une solution, nous n'avons donc pas poursuivi nos tentatives de résolutions par ce solveur.

Le deuxième solveur que nous avons utilisé ne tient pas compte des contraintes de connexité, le modèle repose donc sur une combinatoire beaucoup moins lourde que le premier et nous devrions pouvoir trouver des solutions pour des problèmes de taille plus importante.\\
Cependant, ne tenant pas compte de toutes les contraintes du problème, les solutions données par cet algorithme consitituent donc des bornes supérieures aux solutions optimales.


\begin{table}[h!]
\begin{center}
\begin{tabular}{|l|c|c|c|c|c|c|c|c|c|c|}
\hline
Instances & 1 & 2 & 3 & 4 & 5 & 6 & 7 & 8 & 9 & 10  \\
\hline
5-8 (score)  & 15 & 4 & N/A & 18 & 25 & 13 & 18 & 14 & 9 & 16 \\
\hline
5-8 (temps en s)  & 0.0 & 0.0 & N/A & 0.0 & 0.0 & 0.0 & 0.0 & 0.0 & 0.0 & 0.0 \\
\hline
12-10 (score)       & 42  & 4   & 65  & 62  & 107 & 51  & 56  & 78  & 44  & 115\\
\hline
12-10 (temps en s)  & 6.0 & 6.0 & 2.0 & 1.0 & à.0 & 2.0 & 6.0 & 3.0 & 6.0 & 0.0 \\
\hline
\end{tabular}
\end{center}
\caption{Résultats du solveur Cplex sans connexité}
\end{table}

A partir des instances de taille 15-17, le solveur Cplex utilisant la modélisation sans les contraintes de connexité met plus d'une heure à trouver une solution, nous n'avons donc pas poursuivi nos tentatives de résolutions par ce solveur.

\subsection{Résolution par des heuristiques}

\subsubsection{Méthode Gloutonne Connexe}

Objectif de la méthode gloutonne : obtenir très rapidement une borne inferieure de qualité relativement bonne.

Descrption de la méthode : 

\begin{enumerate}
\item On commence par trier les différentes mailles de l'instance en fonction de la somme de leurs coefficients de handicap.
\item Ensuite on sélectionne une maille de départ parmi celles ayant le plus grand coefficient de handicap pour débuter notre heuristique gloutonne. Dans le même temps on met à jour la somme des coefficient choisi qui doit être 
\item On récupère les voisins de notre maille de départ à la condition que ces voisins soient toujours des mailles de l'instance
\item On sélectionne le voisin de notre maille de départ qui a le coefficient le plus élevé à condition que son ajout dans la liste des mailles sélectionnées ne fasse pas tomber la somme $H^{p}+H^{a}$ en dessous de 2. Puis on supprime cette maille de l'ensemble des voisins et on y ajoute ses voisins.
\item Si l'ajout de cette maille fait tomber la somme $H^{p}+H^{a}$ en dessous de 2, on marque cette maille comme non-sélectionnable et on n'ajoute pas ses voisins à l'ensemble des voisins. 
\item On répète les étapes 4 et 5 en cherchant le voisin de coefficient le plus élevé parmi tous les voisins présents dans notre ensemble de voisins jusqu'à avoir épuisé cette liste.
\item On répète ensuite les opérations 2 à 6 en changeant la maille de départ (parmi la liste des mailles ayant le plus grand coefficient de handicap). On conseve la meilleure solution.
\item La solution finale est connexe, on a donc une borne inférieure de la solution optimale.
\end{enumerate}

\begin{table}[h!]
\begin{center}
\begin{tabular}{|l|c|c|c|c|c|c|c|c|c|c|}
\hline
Instances & 1 & 2 & 3 & 4 & 5 & 6 & 7 & 8 & 9 & 10  \\
\hline
5-8 (score)         & 15  & 4   & N/A & 18  & 25  & 13  & 18  & 14  & 9   & 16 \\
\hline
5-8 (temps en s)    & 0.0 & 0.0 & N/A & 0.0 & 0.0 & 0.0 & 0.0 & 0.0 & 0.0 & 0.0 \\
\hline
12-10 (score)       & 41  & 4   & 55  & 37  & 107 & 50  & 22  & 78  & 32  & 115\\
\hline
12-10 (temps en s)  & 0.0 & 0.0 & 0.0 & 0.0 & 0.0 & 0.0 & 0.0 & 0.0 & 0.0 & 0.0 \\
\hline
15-17 (score)       & 31  & 8   & 102 & 74  & 215 & 68  & 51  & 139 & 89  & 242\\
\hline
15-17 (temps en s)  & 0.0 & 0.0 & 0.0 & 0.0 & 0.0 & 0.0 & 0.0 & 0.0 & 0.0 & 0.0 \\
\hline
20-25 (score)       & 52  & 20  & 158 & 131 & 389 & 69  & 68  & 253 & 189 & 427\\
\hline
20-25 (temps en s)  & 0.0 & 0.0 & 0.0 & 0.0 & 0.0 & 0.0 & 0.0 & 0.0 & 0.0 & 0.0 \\
\hline
25-30 (score)       & 88  & 51  & 229 & 56  & 513 & 79  & 83  & 320 & 254 & 557\\
\hline
25-30 (temps en s)  & 0.0 & 0.0 & 0.0 & 0.0 & 0.0 & 0.0 & 0.0 & 0.0 & 0.0 & 0.0 \\
\hline
\end{tabular}
\end{center}
\caption{Résultats du solveur Glouton Connexe}
\end{table}

Ce solveur nous permet donc de trouver des bornes inférieures de bonne qualité pour chaque instance quasiment instantanément.

\subsubsection{Méthode Gloutonne Non Connexe}

Objectif de la méthode gloutonne non connexe : obtenir très rapidement une borne supérieure de qualité relativement bonne.

\subsubsection{Connexification des résultats de la Méthode Gloutonne Non Connexe}

\subsubsection{Recuit Simulé}

\subsection{Résolution Exacte}

\subsubsection{Utilisation des bornes trouvées grâce aux heuristiques}











\end{document}